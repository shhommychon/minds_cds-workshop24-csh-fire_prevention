\documentclass[
    % Set the default font size, options include: 
    %   8pt, 9pt, 10pt, 11pt, 12pt, 14pt, 17pt, 20pt
    11pt, 
    %
    % Uncomment to set the aspect ratio to a 16:9 ratio which matches
    %   the aspect ratio of 1080p and 4K screens and projectors
    aspectratio=169, 
]{beamer}

\input{template}

\usepackage{booktabs}               % allows the use of \toprule, \midrule and \bottomrule for better rules in tables

% \usepackage{appendixnumberbeamer}   % if you want a separate slide counter for your appendix

% \usepackage{kotex}                  % enable Korean
% \usepackage[utf8]{inputenc}         % allow utf-8 input
% \usepackage[T1]{fontenc}            % use 8-bit T1 fonts
% \usepackage{hyperref}               % hyperlinks
% \usepackage{url}                    % simple URL typesetting
% \usepackage{booktabs}               % professional-quality tables
% \usepackage{amsfonts}               % blackboard math symbols
% \usepackage{amsmath}                % advanced mathematical typesetting
% \usepackage{nicefrac}               % compact symbols for 1/2, etc.
% \usepackage{microtype}              % microtypography

% \usepackage{xcolor}                 % color
% \definecolor{cardinal}{rgb}{0.77, 0.12, 0.23}
% \definecolor{ceruleanblue}{rgb}{0.16, 0.32, 0.75}
% \definecolor{cadmiumgreen}{rgb}{0.0, 0.42, 0.24}

% \usepackage{graphicx}       % enhanced support for graphics
\graphicspath{ {./images/} }  % specifies where to look for included images
% \usepackage{float}          % prevent a figure from moving to a new page

%---------------------------------------------------------
%	PRESENTATION INFORMATION
%---------------------------------------------------------

\title[Middle Footer]{Whisper \& ChatGPT API}
\subtitle{Learning in Humans and Machines: 2nd MINDS-CDS Joint Workshop}
\author[Left Footer]{Sung-Hyu Chon 전성휴}

\institute[]{Mathematical Institute for Data Science, GSAI of POSTECH \\ \smallskip \textit{shhchon@postech.ac.kr}}
\date[Thursday, December 12, 2024]
%\date[\today]

\logo{\includegraphics[width=2.5cm]{minds_logo.png}}

%---------------------------------------------------------
%---------------------------------------------------------
%---------------------------------------------------------

\begin{document}

%---------------------------------------------------------
%	TITLE SLIDE
%---------------------------------------------------------
\section{}
\begin{frame}
	\titlepage % Output the title slide, automatically created using the text entered in the PRESENTATION INFORMATION block above
 
\end{frame}

%---------------------------------------------------------
%	TABLE OF CONTENTS SLIDE
%---------------------------------------------------------
% The table of contents outputs the sections and subsections that appear in your presentation, specified with the standard \section and \subsection commands. You may either display all sections and subsections on one slide with \tableofcontents, or display each section at a time on subsequent slides with \tableofcontents[pausesections]. The latter is useful if you want to step through each section and mention what you will discuss.

\begin{frame}
	\frametitle{Table of Contents} % Slide title, remove this command for no title
	
	\tableofcontents % Output the table of contents (all sections on one slide)
	%\tableofcontents[pausesections] % Output the table of contents (break sections up across separate slides)
\end{frame}

%---------------------------------------------------------
%	PRESENTATION BODY SLIDES
%---------------------------------------------------------
\section{Motivation} % Sections are added in order to organize your presentation into discrete blocks, all sections and subsections are automatically output to the table of contents as an overview of the talk but NOT output in the presentation as separate slides

%------------------------------------------------
\begin{frame}
	\frametitle{Title}
            \emph{To use this template, you can copy and just edit/add slides!}\newline

            This is because all the color customization occurs in the "Customize Themes" section in lines 12-51 of the code\newline

            The remainder of these slides serve an example to show all the features you can use: bullets, buttons, sections, etc.
\end{frame}

%------------------------------------------------
\begin{frame}
	\frametitle{Another Title}
	\framesubtitle{and a subtitle!}
	
	Look at the code of this slide to see how columns made this formatting look nice.

	\begin{columns}[t] % The "c" option specifies centered vertical alignment while the "t" option is used for top vertical alignment
		\begin{column}{0.5\textwidth} % Right column width
                \begin{figure}[h!]
                    \centering
                    %\caption{Hawkins et al, 2015}
                    \includegraphics[angle=0, width=4.5cm]{example1.png}
                    %\label{Figure 1}
                \end{figure}
		\end{column}
  		\begin{column}{0.5\textwidth} % Left column width
                \begin{figure}[h!]
                    \centering
                    %\caption{Hawkins et al, 2015}
                    \includegraphics[angle=0, width=4.5cm]{example1.png}
                    %\label{Figure 1}
                \end{figure}
		\end{column}		
	\end{columns}
\end{frame}

%------------------------------------------------
\begin{frame}
	\frametitle{Yet another title}
            You can use bullets too:\newline
            \begin{itemize}
                \item Like this one\newline
                \item \& this one
            \end{itemize}
\end{frame}

%------------------------------------------------
\begin{frame}
\label{Test} %For the link button for the Appendix slide
	\frametitle{A title}

        \begin{itemize}
            \item You can also nest sub-bullets
            \begin{itemize}
                \item Constant
                \item Cubic
                \item Continuous (with slope changes)
                \item Exponential \newline
            \end{itemize}
        \end{itemize}

        \textbf{Below is a button that links to a slide in the appendix}
        
        \begin{center}
            \hyperlink{Figure}{\beamergotobutton{Go to graphs}}    
        \end{center}
\end{frame}

%------------------------------------------------
\section{Theory}

%------------------------------------------------
\begin{frame}
\label{NeRF Def}
	\frametitle{Frametitle}
	Represent a continuous scene as a 5D vector-valued function $F : (\mathbf{x}, \mathbf{d}) \rightarrow : (\mathbf{c}, \sigma)$ 	
        Here is a made up equation:
        $$ \hat{A} = \bar{m}-\hat{m}_S$$ \newline

        Notice how these buttons are centered and evenly spread out:\newline

        \begin{columns}[t] % The "c" option specifies centered vertical alignment while the "t" option is used for top vertical alignment
		\begin{column}{0.25\textwidth} % Right column width
                \hyperlink{Terms}{\beamergotobutton{Go to Terms}}
		\end{column}
  		\begin{column}{0.25\textwidth} % Left column width
                \hyperlink{Definitions}{\beamergotobutton{Go to Definitions}}
		\end{column}
            \begin{column}{0.25\textwidth} % Left column width
                \hyperlink{Theorems}{\beamergotobutton{Go to Theorems}}
		\end{column}
	\end{columns}
        
\end{frame}

%------------------------------------------------
\begin{frame}
	\frametitle{No way, another title!}
            \begin{enumerate}
                \item Instead of bullets, you can index by number too\newline
                \item like this
            \end{enumerate}
\end{frame}

%------------------------------------------------
\section{Testing}

%------------------------------------------------
\begin{frame}
	\frametitle{Second to last title}
    	\begin{block}{Block Title}
    		Block 1
    	\end{block}
    	
    	\begin{exampleblock}{Example Block Title}
    		Block 2
    	\end{exampleblock}
    	
    	\begin{alertblock}{Alert Block Title}
    		Block 3
    	\end{alertblock}
    	
    	\begin{block}{} % Block without title
    		Block without a title
    	\end{block}
\end{frame}

%------------------------------------------------
\section{Conclusion}
%------------------------------------------------
\begin{frame}
	\frametitle{Last title}
		
	Last bit of text

\end{frame}

%---------------------------------------------------------
%	CLOSING SLIDE
%---------------------------------------------------------

% To remove miniframe from top
\appendix
\setbeamertemplate{headline}{}
\addtobeamertemplate{frametitle}{\vspace*{-\headheight}}{}

\begin{frame}[noframenumbering] %So the end and appendix slides don't contribute to the page count
%[plain] % The optional argument 'plain' hides the headline and footline
	%\frametitle{Questions?}

	\begin{center}
            {\LARGE Questions?}
	\end{center}
 
\end{frame}
%---------------------------------------------------------

%------------------------------------------------
\begin{frame}[noframenumbering]
\label{Figure}
	\frametitle{Appendix - A figure}
        \hyperlink{Test}{\beamerreturnbutton{Return to presentation}}
        
        \begin{figure}[h!]
            \centering
            %\caption{}
            \includegraphics[angle=0, width=5cm]{example2.png}
            %\label{fig}
        \end{figure}
\end{frame}

%------------------------------------------------
\begin{frame}[noframenumbering]
\label{Terms}
	\frametitle{Appendix - Terms}

        \begin{columns}[t] % The "c" option specifies centered vertical alignment while the "t" option is used for top vertical alignment
		\begin{column}{0.5\textwidth} % Right column width
                Some Estimators:
                \begin{itemize}
                    \item Drift: $\hat{\delta}$
                    \item Boundary: $\hat{b}(t)$
                \end{itemize}
		\end{column}
  		\begin{column}{0.5\textwidth} % Left column width
                Some Variables:
                \begin{itemize}
                    \item $\hat{V}$
                    \item $\hat{m}_S$
                    \item $\bar{m}$
                    \item $m_J(\tau)$\newline\newline
                \end{itemize}
		\end{column}
	\end{columns}
        \hyperlink{Test Stat}{\beamerreturnbutton{Return to presentation}}        
\end{frame}

%------------------------------------------------
\begin{frame}[noframenumbering]
\label{Definitions}
	\frametitle{Appendix - Definitions}
         \begin{enumerate}
             \item A definition \newline
         \end{enumerate}
         
        \hyperlink{Test Stat}{\beamerreturnbutton{Return to presentation}}
\end{frame}

%------------------------------------------------
\begin{frame}[noframenumbering]
\label{Theorems}
	\frametitle{Appendix - Theorems}
         \begin{enumerate}
             \item A theorem\newline
         \end{enumerate}
         
        \hyperlink{Test Stat}{\beamerreturnbutton{Return to presentation}}        
\end{frame}

\end{document} 